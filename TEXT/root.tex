\documentclass[11pt,notitlepage,openany]{book}
\usepackage{isabelle,isabellesym,eufrak}
\usepackage[english]{babel}
\usepackage[svgnames]{xcolor}
\usepackage{amsmath, amsthm, amssymb}
\usepackage{listingsutf8, amstext, wasysym}
\usepackage{isabelle-listings}

% For graphics files
%\usepackage[pdftex]{graphicx}
\usepackage[framemethod=default]{mdframed}
\mdfdefinestyle{exampledefault}{%
linecolor=blue, linewidth=2pt,%
rightline=true, leftmargin=0.3cm,
topline=false, bottomline=false, rightline=false}

\newtheorem{thm}{Theorem}[section]
\newtheorem{corr}[thm]{Corollary}
\newtheorem{lemma}[thm]{Lemma}
\newtheorem{prop}[thm]{Lemma}

\theoremstyle{definition}
\newtheorem{defn}[thm]{Definition}

% \begin{thm}\label{...} .... \end{thm}
% \begin{corr}\label{...}  .... \end{cor}
% \begin{lemma}\label{...}  .... \end{lem}
% \begin{prop}\label{...}  .... \end{lem}

\newenvironment{hartshorne}%
  {\begin{mdframed}[style=exampledefault]}%
  {\end{mdframed}}%
%

\newcommand{\term}[1]{{\textbf \emph{#1} }}%


\newcommand{\spike}  {{~\newline \color{red}\rule{1cm}{0.2cm} }}
\newcommand{\david}  {{~\\ \color{brown}\rule{1cm}{0.2cm} }}
\newcommand{\ken}    {{~\\ \color{green}\rule{1cm}{0.2cm} }}
\newcommand{\jackson}{{~\\ \color{lime}\rule{1cm}{0.2cm} }}
\newcommand{\daniel} {{~\newline \color{magenta}\rule{1cm}{0.2cm} }}
\newcommand{\james}  {{~\\ \color{violet}\rule{1cm}{0.2cm} }}
\newcommand{\justin} {{~\\ \color{red}\rule{1cm}{0.2cm} }}
\newcommand{\petar}  {{~\\ \color{LightSalmon}\rule{1cm}{0.2cm} }}
\newcommand{\seiji}  {{~\\ \color{DeepSkyBlue}\rule{1cm}{0.2cm} }}
\newcommand{\caleb}  {{~\\ \color{BlueViolet}\rule{1cm}{0.2cm} }}
\newcommand{\homer}  {{~\\ \color{orange}\rule{1cm}{0.2cm} }}
\newcommand{\siqi}   {{~\\ \color{cyan}\rule{1cm}{0.2cm} }}
\newcommand{\haoze}  {{~\\ \color{blue}\rule{1cm}{0.2cm} }}
\newcommand{\done}   {{~\newline \color{black}\rule{1cm}{0.2cm} }}



% this should be the last package used
\usepackage{pdfsetup}

% urls in roman style, theory text in math-similar italics
\urlstyle{rm}
\isabellestyle{it}

\begin{document}

\title{Foundations of Projective Geometry: A formalization}
\author{John Hughes and the students of Brown CS1951D, Spring 2020}
\maketitle

%\begin{abstract}
%Starting from Robin Hartshorne's book, we prove things.
%\end{abstract}

% \tableofcontents

\chapter*{Introduction}
This text is a formalization of Robin Hartshorne's \emph{Foundations of Projective Geometry}
using the Isabelle proof assistant, primarily relying on Isar. Quotations 
from Hartshorne appear indented, with a blue line to the left. Additional material 
written by either the professor (John (Spike) Hughes) or various students are surrounded by colored 
markers, like this:
\spike
This is something written by Spike
\done
with the black marker indicating the end of the section written by Spike (except that in this case, 
it's part of a larger section Spike wrote). 

Within Isabelle, numbered propositions or theorems from Hartshorne are given names that tie back 
to the text, so Proposition 1.1 in the text is called \texttt{Prop1P1}, with ``P'' replacing the period, 
for instance. 

Students should insert things into the document using
macros associated with their name (in lowercase) to produce a marker indicating the start of their contribution, and the "done" macro to indicate when their contribution is complete. Here are examples of the macro results:

\spike Spike
\david David
\ken Ken
\jackson Jackson
\daniel Daniel
\brad Brad
\justin Justin
\petar Petar
\seiji Seiji
\caleb Caleb
\homer Homer
\siqi Siqi
\haoze Haoze
\done the ``done'' macro.


%
\begin{isabellebody}%
\setisabellecontext{IbookCh{\isadigit{0}}}%
%
\isadelimtheory
%
\endisadelimtheory
%
\isatagtheory
\isacommand{theory}\isamarkupfalse%
\ IbookCh{\isadigit{0}}\isanewline
\ \ \isakeyword{imports}\ Main\isanewline
\isakeyword{begin}%
\endisatagtheory
{\isafoldtheory}%
%
\isadelimtheory
\isanewline
%
\endisadelimtheory
\isanewline
\isacommand{lemma}\isamarkupfalse%
\ {\isachardoublequoteopen}evens{\isachardoublequoteclose}{\isacharcolon}{\kern0pt}\ {\isachardoublequoteopen}{\isasymexists}\ {\isacharparenleft}{\kern0pt}n{\isacharcolon}{\kern0pt}{\isacharcolon}{\kern0pt}nat{\isacharparenright}{\kern0pt}\ {\isachardot}{\kern0pt}\ {\isadigit{2}}{\isacharasterisk}{\kern0pt}n\ {\isachargreater}{\kern0pt}\ {\isacharparenleft}{\kern0pt}k{\isacharcolon}{\kern0pt}{\isacharcolon}{\kern0pt}nat{\isacharparenright}{\kern0pt}{\isachardoublequoteclose}\isanewline
%
\isadelimproof
\ \ %
\endisadelimproof
%
\isatagproof
\isacommand{by}\isamarkupfalse%
\ presburger%
\endisatagproof
{\isafoldproof}%
%
\isadelimproof
\isanewline
%
\endisadelimproof
\isanewline
\isacommand{lemma}\isamarkupfalse%
\ {\isachardoublequoteopen}evens{\isadigit{2}}{\isachardoublequoteclose}{\isacharcolon}{\kern0pt}\ {\isachardoublequoteopen}{\isasymexists}\ {\isacharparenleft}{\kern0pt}n{\isacharcolon}{\kern0pt}{\isacharcolon}{\kern0pt}nat{\isacharparenright}{\kern0pt}\ {\isachardot}{\kern0pt}\ {\isadigit{2}}{\isacharasterisk}{\kern0pt}n\ {\isachargreater}{\kern0pt}\ {\isacharparenleft}{\kern0pt}k{\isacharcolon}{\kern0pt}{\isacharcolon}{\kern0pt}nat{\isacharparenright}{\kern0pt}{\isachardoublequoteclose}\isanewline
%
\isadelimproof
\ \ %
\endisadelimproof
%
\isatagproof
\isacommand{using}\isamarkupfalse%
\ evens\ \isacommand{by}\isamarkupfalse%
\ auto%
\endisatagproof
{\isafoldproof}%
%
\isadelimproof
\isanewline
%
\endisadelimproof
\isanewline
\isacommand{lemma}\isamarkupfalse%
\ {\isachardoublequoteopen}evens{\isadigit{3}}{\isachardoublequoteclose}{\isacharcolon}{\kern0pt}\ {\isachardoublequoteopen}{\isasymexists}\ {\isacharparenleft}{\kern0pt}n{\isacharcolon}{\kern0pt}{\isacharcolon}{\kern0pt}nat{\isacharparenright}{\kern0pt}\ {\isachardot}{\kern0pt}\ {\isadigit{2}}{\isacharasterisk}{\kern0pt}n\ {\isachargreater}{\kern0pt}\ {\isacharparenleft}{\kern0pt}k{\isacharcolon}{\kern0pt}{\isacharcolon}{\kern0pt}nat{\isacharparenright}{\kern0pt}{\isachardoublequoteclose}\isanewline
%
\isadelimproof
%
\endisadelimproof
%
\isatagproof
\isacommand{proof}\isamarkupfalse%
\ {\isacharminus}{\kern0pt}\isanewline
\ \ \isacommand{have}\isamarkupfalse%
\ example{\isacharcolon}{\kern0pt}{\isachardoublequoteopen}{\isadigit{2}}{\isacharasterisk}{\kern0pt}{\isacharparenleft}{\kern0pt}k{\isacharplus}{\kern0pt}{\isadigit{1}}{\isacharparenright}{\kern0pt}\ {\isachargreater}{\kern0pt}\ k{\isachardoublequoteclose}\ \isanewline
\ \ \ \ \isacommand{by}\isamarkupfalse%
\ simp\isanewline
\ \ \isacommand{show}\isamarkupfalse%
\ {\isacharquery}{\kern0pt}thesis\ \isanewline
\ \ \ \ \isacommand{using}\isamarkupfalse%
\ example\ \isacommand{by}\isamarkupfalse%
\ blast\isanewline
\isacommand{qed}\isamarkupfalse%
%
\endisatagproof
{\isafoldproof}%
%
\isadelimproof
\isanewline
%
\endisadelimproof
\isanewline
\isacommand{lemma}\isamarkupfalse%
\ {\isachardoublequoteopen}evens{\isadigit{4}}{\isachardoublequoteclose}{\isacharcolon}{\kern0pt}\ {\isachardoublequoteopen}{\isasymexists}\ {\isacharparenleft}{\kern0pt}n{\isacharcolon}{\kern0pt}{\isacharcolon}{\kern0pt}nat{\isacharparenright}{\kern0pt}\ {\isachardot}{\kern0pt}\ {\isadigit{2}}{\isacharasterisk}{\kern0pt}n\ {\isachargreater}{\kern0pt}\ {\isacharparenleft}{\kern0pt}k{\isacharcolon}{\kern0pt}{\isacharcolon}{\kern0pt}nat{\isacharparenright}{\kern0pt}{\isachardoublequoteclose}\isanewline
%
\isadelimproof
%
\endisadelimproof
%
\isatagproof
\isacommand{proof}\isamarkupfalse%
\ {\isacharparenleft}{\kern0pt}cases\ {\isachardoublequoteopen}k\ {\isacharequal}{\kern0pt}\ {\isadigit{0}}{\isachardoublequoteclose}{\isacharparenright}{\kern0pt}\ \isanewline
\ \ \isacommand{case}\isamarkupfalse%
\ True\ \ \ \ \ \ \ \ \ \ \ \isanewline
\ \ \isacommand{have}\isamarkupfalse%
\ ex{\isacharcolon}{\kern0pt}\ {\isachardoublequoteopen}\ {\isadigit{2}}\ {\isacharasterisk}{\kern0pt}\ {\isadigit{1}}\ {\isachargreater}{\kern0pt}\ k{\isachardoublequoteclose}\ \isacommand{using}\isamarkupfalse%
\ True\ \isacommand{by}\isamarkupfalse%
\ simp\ \ \isanewline
\ \ \isacommand{then}\isamarkupfalse%
\ \isacommand{show}\isamarkupfalse%
\ {\isacharquery}{\kern0pt}thesis\ \isacommand{using}\isamarkupfalse%
\ ex\ \isacommand{by}\isamarkupfalse%
\ blast\isanewline
\isacommand{next}\isamarkupfalse%
\isanewline
\ \ \isacommand{case}\isamarkupfalse%
\ False\ \isanewline
\ \ \isacommand{have}\isamarkupfalse%
\ ex{\isacharcolon}{\kern0pt}\ {\isachardoublequoteopen}{\isadigit{2}}\ {\isacharasterisk}{\kern0pt}\ k\ {\isachargreater}{\kern0pt}\ k{\isachardoublequoteclose}\ \isacommand{using}\isamarkupfalse%
\ False\ \isacommand{by}\isamarkupfalse%
\ auto\ \isanewline
\ \ \isacommand{then}\isamarkupfalse%
\ \isacommand{show}\isamarkupfalse%
\ {\isacharquery}{\kern0pt}thesis\ \isacommand{using}\isamarkupfalse%
\ ex\ \isacommand{by}\isamarkupfalse%
\ blast\ \isanewline
\isacommand{qed}\isamarkupfalse%
%
\endisatagproof
{\isafoldproof}%
%
\isadelimproof
\isanewline
%
\endisadelimproof
\isanewline
\isanewline
\isanewline
\isacommand{lemma}\isamarkupfalse%
\ {\isachardoublequoteopen}evens{\isadigit{5}}{\isachardoublequoteclose}{\isacharcolon}{\kern0pt}\ {\isachardoublequoteopen}\ {\isasymexists}\ {\isacharparenleft}{\kern0pt}n{\isacharcolon}{\kern0pt}{\isacharcolon}{\kern0pt}nat{\isacharparenright}{\kern0pt}\ {\isachardot}{\kern0pt}\ {\isadigit{2}}{\isacharasterisk}{\kern0pt}n\ {\isachargreater}{\kern0pt}\ {\isacharparenleft}{\kern0pt}k{\isacharcolon}{\kern0pt}{\isacharcolon}{\kern0pt}nat{\isacharparenright}{\kern0pt}{\isachardoublequoteclose}\isanewline
%
\isadelimproof
%
\endisadelimproof
%
\isatagproof
\isacommand{proof}\isamarkupfalse%
\ {\isacharparenleft}{\kern0pt}induction\ k{\isacharparenright}{\kern0pt}\isanewline
\ \ \isacommand{case}\isamarkupfalse%
\ {\isadigit{0}}\isanewline
\ \ \isacommand{then}\isamarkupfalse%
\ \isacommand{show}\isamarkupfalse%
\ {\isacharquery}{\kern0pt}case\ \isacommand{by}\isamarkupfalse%
\ auto\isanewline
\isacommand{next}\isamarkupfalse%
\isanewline
\ \ \isacommand{case}\isamarkupfalse%
\ {\isacharparenleft}{\kern0pt}Suc\ k{\isacharparenright}{\kern0pt}\isanewline
\ \ \isacommand{then}\isamarkupfalse%
\ \isacommand{obtain}\isamarkupfalse%
\ n\ \isakeyword{where}\ nFact{\isacharcolon}{\kern0pt}\ {\isachardoublequoteopen}{\isadigit{2}}{\isacharasterisk}{\kern0pt}n\ {\isachargreater}{\kern0pt}\ k{\isachardoublequoteclose}\ \isacommand{by}\isamarkupfalse%
\ blast\isanewline
\ \ \isacommand{let}\isamarkupfalse%
\ {\isacharquery}{\kern0pt}m\ {\isacharequal}{\kern0pt}\ {\isachardoublequoteopen}Suc\ n{\isachardoublequoteclose}\ \isanewline
\ \ \isacommand{from}\isamarkupfalse%
\ nFact\ \isacommand{have}\isamarkupfalse%
\ {\isachardoublequoteopen}{\isadigit{2}}\ {\isacharasterisk}{\kern0pt}\ {\isacharquery}{\kern0pt}m\ {\isachargreater}{\kern0pt}\ Suc\ k{\isachardoublequoteclose}\ \isacommand{by}\isamarkupfalse%
\ auto\isanewline
\ \ \isacommand{then}\isamarkupfalse%
\ \isacommand{show}\isamarkupfalse%
\ {\isacharquery}{\kern0pt}case\ \isacommand{by}\isamarkupfalse%
\ blast\isanewline
\isacommand{qed}\isamarkupfalse%
%
\endisatagproof
{\isafoldproof}%
%
\isadelimproof
\isanewline
%
\endisadelimproof
\isanewline
\isanewline
\isacommand{lemma}\isamarkupfalse%
\ {\isachardoublequoteopen}T{\isadigit{5}}{\isachardoublequoteclose}{\isacharcolon}{\kern0pt}\ {\isachardoublequoteopen}\ {\isasymexists}\ {\isacharparenleft}{\kern0pt}n{\isacharcolon}{\kern0pt}{\isacharcolon}{\kern0pt}nat{\isacharparenright}{\kern0pt}\ {\isachardot}{\kern0pt}\ {\isacharparenleft}{\kern0pt}k{\isacharcolon}{\kern0pt}{\isacharcolon}{\kern0pt}nat{\isacharparenright}{\kern0pt}\ {\isacharplus}{\kern0pt}\ {\isadigit{5}}\ {\isacharless}{\kern0pt}\ n{\isachardoublequoteclose}\isanewline
%
\isadelimproof
%
\endisadelimproof
%
\isatagproof
\isacommand{proof}\isamarkupfalse%
\ {\isacharparenleft}{\kern0pt}induction\ k{\isacharparenright}{\kern0pt}\isanewline
\ \ \isacommand{case}\isamarkupfalse%
\ {\isadigit{0}}\isanewline
\ \ \isacommand{then}\isamarkupfalse%
\ \isacommand{show}\isamarkupfalse%
\ {\isacharquery}{\kern0pt}case\ \isacommand{by}\isamarkupfalse%
\ auto\isanewline
\isacommand{next}\isamarkupfalse%
\isanewline
\ \ \isacommand{case}\isamarkupfalse%
\ {\isacharparenleft}{\kern0pt}Suc\ k{\isacharparenright}{\kern0pt}\isanewline
\ \ \isacommand{then}\isamarkupfalse%
\ \isacommand{obtain}\isamarkupfalse%
\ n\ \isakeyword{where}\ n{\isacharcolon}{\kern0pt}\ {\isacartoucheopen}k\ {\isacharplus}{\kern0pt}\ {\isadigit{5}}\ {\isacharless}{\kern0pt}\ n{\isacartoucheclose}\ %
\isamarkupcmt{From this, I'd like to say "OK, so there's some n with that property. Let's call it m.%
}\isanewline
\ \ \ \ \isacommand{by}\isamarkupfalse%
\ blast\isanewline
\ \ \isacommand{let}\isamarkupfalse%
\ {\isacharquery}{\kern0pt}p\ {\isacharequal}{\kern0pt}\ {\isacartoucheopen}Suc\ n{\isacartoucheclose}\isanewline
\ \ \isacommand{from}\isamarkupfalse%
\ n\ \isacommand{have}\isamarkupfalse%
\ {\isacartoucheopen}Suc\ k\ {\isacharplus}{\kern0pt}\ {\isadigit{5}}\ {\isacharless}{\kern0pt}\ {\isacharquery}{\kern0pt}p{\isacartoucheclose}\ %
\isamarkupcmt{I show that Suc k + 5 < p for some p (as I did in the k = 0 case)%
}\isanewline
\ \ \ \ \isacommand{by}\isamarkupfalse%
\ auto\isanewline
\ \ \isacommand{then}\isamarkupfalse%
\ \isacommand{show}\isamarkupfalse%
\ {\isacharquery}{\kern0pt}case\ \isacommand{try{\isadigit{0}}}\isamarkupfalse%
\isanewline
\ \ \ \ \isacommand{by}\isamarkupfalse%
\ {\isacharparenleft}{\kern0pt}rule\ exI{\isacharbrackleft}{\kern0pt}of\ {\isacharunderscore}{\kern0pt}\ {\isacartoucheopen}Suc\ n{\isacartoucheclose}{\isacharbrackright}{\kern0pt}{\isacharparenright}{\kern0pt}\ %
\isamarkupcmt{Isabelle can infer the "there-exists" result from that instance, using P(a) => EX x . P(x).%
}\isanewline
\isacommand{qed}\isamarkupfalse%
%
\endisatagproof
{\isafoldproof}%
%
\isadelimproof
\isanewline
%
\endisadelimproof
\isanewline
\isanewline
\isanewline
\isacommand{definition}\isamarkupfalse%
\ nand\ {\isacharcolon}{\kern0pt}{\isacharcolon}{\kern0pt}\ {\isachardoublequoteopen}bool\ {\isasymRightarrow}\ bool\ {\isasymRightarrow}\ bool{\isachardoublequoteclose}\ {\isacharparenleft}{\kern0pt}\isakeyword{infixr}\ {\isachardoublequoteopen}N{\isachardoublequoteclose}\ {\isadigit{3}}{\isadigit{5}}{\isacharparenright}{\kern0pt}\isanewline
\ \ \isakeyword{where}\ {\isachardoublequoteopen}A\ N\ B\ {\isasymequiv}\ {\isasymnot}{\isacharparenleft}{\kern0pt}A\ {\isasymand}\ B{\isacharparenright}{\kern0pt}{\isachardoublequoteclose}\isanewline
\isanewline
\isacommand{lemma}\isamarkupfalse%
\ {\isachardoublequoteopen}\ {\isasymnot}A\ {\isasymlongleftrightarrow}\ A\ N\ A{\isachardoublequoteclose}\isanewline
%
\isadelimproof
\ \ %
\endisadelimproof
%
\isatagproof
\isacommand{unfolding}\isamarkupfalse%
\ nand{\isacharunderscore}{\kern0pt}def\isanewline
\ \ \isacommand{apply}\isamarkupfalse%
\ simp\isanewline
\ \ \isacommand{done}\isamarkupfalse%
%
\endisatagproof
{\isafoldproof}%
%
\isadelimproof
\isanewline
%
\endisadelimproof
\isanewline
\isanewline
\isanewline
\isanewline
\isanewline
\isacommand{lemma}\isamarkupfalse%
\ {\isachardoublequoteopen}odds{\isadigit{1}}{\isachardoublequoteclose}{\isacharcolon}{\kern0pt}\ {\isachardoublequoteopen}\ {\isasymexists}\ {\isacharparenleft}{\kern0pt}k{\isacharcolon}{\kern0pt}{\isacharcolon}{\kern0pt}nat{\isacharparenright}{\kern0pt}\ {\isachardot}{\kern0pt}\ {\isadigit{2}}{\isacharasterisk}{\kern0pt}k{\isacharplus}{\kern0pt}{\isadigit{1}}\ {\isachargreater}{\kern0pt}\ {\isacharparenleft}{\kern0pt}n{\isacharcolon}{\kern0pt}{\isacharcolon}{\kern0pt}nat{\isacharparenright}{\kern0pt}{\isachardoublequoteclose}\isanewline
%
\isadelimproof
%
\endisadelimproof
%
\isatagproof
\isacommand{proof}\isamarkupfalse%
\ {\isacharminus}{\kern0pt}\isanewline
\ \ \isacommand{have}\isamarkupfalse%
\ {\isachardoublequoteopen}{\isadigit{2}}{\isacharasterisk}{\kern0pt}n{\isacharplus}{\kern0pt}{\isadigit{1}}\ {\isachargreater}{\kern0pt}\ n{\isachardoublequoteclose}\ \isacommand{by}\isamarkupfalse%
\ auto\isanewline
\ \ \isacommand{thus}\isamarkupfalse%
\ {\isacharquery}{\kern0pt}thesis\ \isacommand{by}\isamarkupfalse%
\ blast\isanewline
\isacommand{qed}\isamarkupfalse%
%
\endisatagproof
{\isafoldproof}%
%
\isadelimproof
\isanewline
%
\endisadelimproof
\isanewline
\isacommand{lemma}\isamarkupfalse%
\ {\isachardoublequoteopen}positiveSquare{\isachardoublequoteclose}{\isacharcolon}{\kern0pt}\ {\isachardoublequoteopen}{\isacharparenleft}{\kern0pt}n{\isacharcolon}{\kern0pt}{\isacharcolon}{\kern0pt}nat{\isacharparenright}{\kern0pt}\ {\isacharasterisk}{\kern0pt}\ {\isacharparenleft}{\kern0pt}n{\isacharcolon}{\kern0pt}{\isacharcolon}{\kern0pt}nat{\isacharparenright}{\kern0pt}\ {\isasymge}\ {\isadigit{0}}{\isachardoublequoteclose}\isanewline
%
\isadelimproof
\ \ %
\endisadelimproof
%
\isatagproof
\isacommand{by}\isamarkupfalse%
\ auto%
\endisatagproof
{\isafoldproof}%
%
\isadelimproof
\isanewline
%
\endisadelimproof
\isanewline
\isanewline
%
\isadelimtheory
\isanewline
%
\endisadelimtheory
%
\isatagtheory
\isacommand{end}\isamarkupfalse%
%
\endisatagtheory
{\isafoldtheory}%
%
\isadelimtheory
%
\endisadelimtheory
%
\end{isabellebody}%
\endinput
%:%file=~/Desktop/IsabelleBook/PROVING/IbookCh0.thy%:%
%:%10=1%:%
%:%11=1%:%
%:%12=2%:%
%:%13=3%:%
%:%18=3%:%
%:%21=4%:%
%:%22=5%:%
%:%23=5%:%
%:%26=6%:%
%:%30=6%:%
%:%31=6%:%
%:%36=6%:%
%:%39=7%:%
%:%40=8%:%
%:%41=8%:%
%:%44=9%:%
%:%48=9%:%
%:%49=9%:%
%:%50=9%:%
%:%55=9%:%
%:%58=10%:%
%:%59=11%:%
%:%60=11%:%
%:%67=12%:%
%:%68=12%:%
%:%69=13%:%
%:%70=13%:%
%:%71=14%:%
%:%72=14%:%
%:%73=15%:%
%:%74=15%:%
%:%75=16%:%
%:%76=16%:%
%:%77=16%:%
%:%78=17%:%
%:%84=17%:%
%:%87=18%:%
%:%88=19%:%
%:%89=19%:%
%:%96=20%:%
%:%97=20%:%
%:%98=21%:%
%:%99=21%:%
%:%100=22%:%
%:%101=22%:%
%:%102=22%:%
%:%103=22%:%
%:%104=23%:%
%:%105=23%:%
%:%106=23%:%
%:%107=23%:%
%:%108=23%:%
%:%109=24%:%
%:%110=24%:%
%:%111=25%:%
%:%112=25%:%
%:%113=26%:%
%:%114=26%:%
%:%115=26%:%
%:%116=26%:%
%:%117=27%:%
%:%118=27%:%
%:%119=27%:%
%:%120=27%:%
%:%121=27%:%
%:%122=28%:%
%:%128=28%:%
%:%131=29%:%
%:%132=33%:%
%:%133=34%:%
%:%134=35%:%
%:%135=35%:%
%:%142=36%:%
%:%143=36%:%
%:%144=37%:%
%:%145=37%:%
%:%146=38%:%
%:%147=38%:%
%:%148=38%:%
%:%149=38%:%
%:%150=39%:%
%:%151=39%:%
%:%152=40%:%
%:%153=40%:%
%:%154=41%:%
%:%155=41%:%
%:%156=41%:%
%:%157=41%:%
%:%158=42%:%
%:%159=42%:%
%:%160=43%:%
%:%161=43%:%
%:%162=43%:%
%:%163=43%:%
%:%164=44%:%
%:%165=44%:%
%:%166=44%:%
%:%167=44%:%
%:%168=45%:%
%:%174=45%:%
%:%177=46%:%
%:%178=47%:%
%:%179=48%:%
%:%180=48%:%
%:%187=49%:%
%:%188=49%:%
%:%189=50%:%
%:%190=50%:%
%:%191=51%:%
%:%192=51%:%
%:%193=51%:%
%:%194=51%:%
%:%195=52%:%
%:%196=52%:%
%:%197=53%:%
%:%198=53%:%
%:%199=54%:%
%:%200=54%:%
%:%201=54%:%
%:%202=54%:%
%:%203=54%:%
%:%204=55%:%
%:%205=55%:%
%:%206=56%:%
%:%207=56%:%
%:%208=57%:%
%:%209=57%:%
%:%210=57%:%
%:%211=57%:%
%:%212=57%:%
%:%213=58%:%
%:%214=58%:%
%:%215=59%:%
%:%216=59%:%
%:%217=59%:%
%:%218=59%:%
%:%219=60%:%
%:%220=60%:%
%:%221=60%:%
%:%222=60%:%
%:%223=61%:%
%:%229=61%:%
%:%232=62%:%
%:%233=63%:%
%:%234=64%:%
%:%235=65%:%
%:%236=65%:%
%:%237=66%:%
%:%238=67%:%
%:%239=68%:%
%:%240=68%:%
%:%243=69%:%
%:%247=69%:%
%:%248=69%:%
%:%249=70%:%
%:%250=70%:%
%:%251=71%:%
%:%257=71%:%
%:%260=72%:%
%:%261=73%:%
%:%262=78%:%
%:%263=79%:%
%:%264=80%:%
%:%265=81%:%
%:%266=81%:%
%:%273=82%:%
%:%274=82%:%
%:%275=83%:%
%:%276=83%:%
%:%277=83%:%
%:%278=84%:%
%:%279=84%:%
%:%280=84%:%
%:%281=85%:%
%:%287=85%:%
%:%290=86%:%
%:%291=87%:%
%:%292=87%:%
%:%295=88%:%
%:%299=88%:%
%:%300=88%:%
%:%305=88%:%
%:%308=89%:%
%:%309=104%:%
%:%312=105%:%
%:%317=106%:%



\bibliographystyle{abbrv}
\bibliography{root}

\end{document}
