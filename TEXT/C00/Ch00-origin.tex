\chapter*{Origin Story}
This text arose from an attempt at a formalization of Robin Hartshorne's \emph{Foundations of Projective Geometry}
using the Isabelle proof assistant, primarily relying on Isar. That project got started in 2018, continued through a small course in 2020 (interrupted by the pandemic), and restarted in 2023. As a former mathematician, I chose Hartshorne's book because of its relative simplicity -- a projective geometry is a far simpler object than, say, a field or a module, and at least in its simplest form doesn't require any algebraic objects at all -- just four simple axioms about a pair of sets. It seemed as if it should be easy to formalize, especially for someone who happened to already know some ML (the language underlying Isabelle) and who was once a mathematician, hence familiar with proofs. 

It wasn't easy after all, but the problem wasn't either the complexity of projective geometries or my experience with ML and mathematics -- it was a lack of a text that told me enough about Isabelle to let me learn it systematically from the point of view of someone who wanted to do mathematics, but had little interest or experience in formal logic. Many of the texts describe how to prove things about \textit{programs}, which is generally interesting, but tends to use a lot of induction. Hartshorne's book, by contrast, uses induction only once, and it's an induction on a sequence of nested structures rather than an induction over some single type or data structure. Many books also take examples from logic, which I found baffling, as I was struggling to understand the difference between Isabelle's `metalogic' and the `object logic', and the simple logic theorems used as illustrations were often confusing to me. 

As I've gradually begun to gather a more systematic understand of Isabelle (with a great deal of help, and some other sets of notes like this one), I've begun writing this book, trying to retain my innocence: a proper degree of confusion about what to do next, something that experienced Isabelle users lack. They say things like ``this is an ordinary co-induction!" or ``prove that `there-exists' proposition by just providing a witness," but they fail to realize how very much is wrapped up in such a telegraphic suggestion. Hence I've written this rather wordy and slow-moving introduction. 
