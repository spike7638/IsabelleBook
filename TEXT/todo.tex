\chapter*{To Do}
\begin{itemize}
\item 
Stuff isabelle produces: what do def, fun, etc, insert into environment. How do simpset change with each "fun", etc. What does "quotient" and "lifting/transfer" produce? 
\item page 31 of \|https://isabelle.in.tum.de/doc/tutorial.pdf|
\item using sledgehammer: put after "using" or "unfolding" or "have ..." or "show ..." but NOT after "by"
\item in sledgehammer, figure out how to turn on detailed feedback; apparently the magic settings are [verbose, debug] .
\item point people to read the first few pages (up to the "options" section) of the sledgehammer manual: https://isabelle.in.tum.de/dist/doc/sledgehammer.pdf
\item Also: a bunch of nice facts here -- https://stackoverflow.com/questions/21694248/isabelle-why-do-i-get-completely-different-results-when-running-try-versus-sled -- that I might want to incorporate. 
\item explain coding: grey background = isabelle code; purple background = isabelle interface components (buttons, files, panels) or interface-responses (the contents of the 'output' panel, for example), except in the rare cases where these are actual isabelle code; peach background = systems things, like filenames typed into dialog boxes. Because such filenames can also be part of isabelle code, I won't be completely consistent about this, but I'll try to at least be sensible.
\item .Fix spacing in isabelle code style.
\item Move style files to subdirectory
\item Make all latex intermediate files be ignored by git
\item add section on cartouches
\item Fix marginnote to be able to contain "isi" stuff.
\item +For the "sys" and "co" languages, make sure that all characters are un=special (e.g., tildes!)
\item Introduce "Of" and "where" in "using" clauses
\item Why is code "verbatim-like" and can this be fixed? 
\item .Restructure into chapters
\item .define displayed-code macro: Perhaps IC for displayed code? 
\item .Figure out image inclusion
\item .get bibliography working
\item .get non-keyword text in Isabelle source to be ttfamily as well (e.g. presburger)
\item .Define digression-box (add "digression" header)
\item .fix quotes in Isabelle code to be stupid quotes
\item .define clean inline-isabelle macro 
\item .Reduce left margin; increase right margin so marginpars can fit.
\item .get Sys language definition working: fixed wdth font, pale blue background
\end{itemize}

\begin{verbatim}

Pointer to isabelle community cookbook: https://isabelle.systems/cookbook


=======
"value" to use a debugging tool (and warning about things whose values can't be printed)
datatype pointD = Hd | Ad | Bd | Cd| Dd| Ed| Fd| Pd | Qd |Rd
definition PointsD::"pointD set"  where "PointsD = {Hd, Ad, Bd, Cd, Dd, Ed, Fd, Pd, Qd, Rd}"

value "PointsD" produces
"{Hd, Ad, Bd, Cd, Dd, Ed, Fd, Pd, Qd, Rd}"
  :: "pointD set"


"thm" as a debugging tool

Every theorem is a function! Example: 
lemma X: 
  assumes "a > 10"
  assumes "b > 15"
  shows "a + b > 25"
  
shows up as 

[?a > 10] => [?b > 15] => [?a + ?b > 25]

That means you can type X [OF 13 19] and get

[13 > 10] => [19 > 15] => [13 + 19 > 25]

which you can read as "if 13 > 10 adn 19 > 15, then 13 + 19 > 25." Cool!

Using x by y : you can see what's happening here by looking at the state with your cursor just before 'using', just after x, etc. 

tracing simp


How do I write a cases proof for trichotomy? 

\end{verbatim}


\subsection{Doing proofs right}
Once you've done some more Spivak proofs ... we'll move into affine geometry. 
To prove the affine plane on 4 points is an affine plane, we need to show all the axioms hold. The only really messy one is 
\begin{verbatim}
    "¬ affine_plane_data.collinear
           A4Points
           A4Lines
           A4meets Pa Qa Ra"
\end{verbatim}
Proving a negation \textit{could} be done by expanding the definition of collinear (a conjunction of several items), negating it (using blast), and then showing that each of the resulting disjunction elements is false...but one of these is existentially quantified, so it becomes universally quantified, so it requires working through all possible lines one at a time. 

Instead, the right approach is classical contradiction: we assume they ARE collinear, fix a line $n$ that contains all three, and show this leads to nonsense. (A simple way, in this case, is to observe that the cardinality of $n$ must be at least 3, but the cardinatlity of each line in A4 is exactly 2, but we instead work through the details because it's instructive for later proofs, and Isabelle is good at enumerate-and-check approaches.) 

Warning: I want to use ccontr to show the statement above. My first version said
\begin{IS}
proof (rule ccontr)
    assume cHyp: "affine_plane_data.collinear
           A4Points
           A4Lines
           A4meets Pa Qa Ra"
    ...
\end{IS
but that failed badly. Why? Because the thing you assume must be the negation of the thing you want to prove; I should have written
\begin{IS}
proof (rule ccontr)
    assume cHyp: "¬¬affine_plane_data.collinear
    ...
\end{IS
i.e., simply put a negation sign in front of the desired goal. The fact that ¬¬ is the same as "no negations at all" eludes Isabelle (!)

Need to discuss sledgehammer, and how it has access to a million facts, but won't always choose the one you need. Nudging it with a helpful "using" (esp. using XXX [of Y Z W]) can be a big help. Better still (according to Eberl) is to use specific tactics

He says this: "-- I look at proofs in HOL-Algebra and ask myself "why aren't my proofs \newline
-- ever that short?"

Like most things, it's a matter of experience. You just learn how to
write things down in ways that are nice and concise and perhaps easiest
for the automation to deal with. And you learn how to use all the
different tactics for the best effect. I for one am a big fan of doing
every step in Isar by first using a very specific, predictable tactic
like "rule", "intro", "subst" and then getting rid of the arising
subgoals with something more hard-hitting and unfocused like "auto". I
do occasionally use sledgehammer, but perhaps not as much as other
people. I tend to not like the kinds of proofs it creates.

--End of Eberl quote --


tactic to learn: locale_unfolding




Example citation: \cite{latex2e}.

\begin{verbatim}
Book design:
Digression: footnote with highlighted background, or margin paragraph
Code: both inline and displayed, in an Isabelle-like format
Topic coloring? Interface, proving, logic, syntax (?)
Chapter summary and high points at end of each chapter


Marginnotes
=====

Homework files: 
ChXX_HWxx_topic.thy
ChXX_HWxx_topic_sol.thy [solutions]

Sample files
ChXX_Sxx_topic.thy
ChXX_Sxx_topic_sol.thy

Code-image-files:
Org by chapter
xxx_topic.png
Xxx.xx_topic.png  (later additions inserted out of sequence)

Scratch files
CHXX_scratchxx

\end{verbatim}


\ignore{ 
Example of a digression:

\digression{Lorem   ipsum. Lorem   ipsum. Lorem   ipsum. Lorem   ipsum. Lorem   ipsum. Lorem   ipsum. Lorem   ipsum. Lorem   ipsum.Lorem   ipsum.Lorem   ipsum.Lorem   ipsum.Lorem   ipsum. }

Example of a digression-with-topic:

\digression[unit measures]{Lorem   ipsum. Lorem   ipsum. Lorem   ipsum. Lorem   ipsum. Lorem   ipsum. Lorem   ipsum. Lorem   ipsum. Lorem   ipsum.Lorem   ipsum.Lorem   ipsum.Lorem   ipsum.Lorem   ipsum. }
And a todo: \todo{remove this}

Example of sys code: \sys{ls ~jfh}.

Example of Inline Isabelle code \isi{theory IbookCh0} and general Isabelle code:
}

"Where do I find a library that does X?" See \verb|https://isabelle.in.tum.de/library/HOL/index.html|

Fix-assume-show notation: you can also include "defines":
\begin{verbatim}
lemma A2_vert5:
  fixes x0 x1 :: real
  defines l_form: ‹l ≡ A2Vertical x0› 
  defines k_form: ‹k ≡ A2Vertical x1›
  assumes ‹a2meets T l› and ‹a2meets T k›
  shows ‹l = k› 
  by (cases T) (use assms in auto)
\end{verbatim}

"How do I create a set using {x | x > 0} notation?" -- see https://isabelle.zulipchat.com/#narrow/channel/238552-Beginner-Questions/topic/Construction.20a.20set

"How should I define my function (with "definition" or "fun" or "abbreviation" perhaps?) and why?" 

"unfolding" vs "using"
Lovely problem for HW2: https://math.stackexchange.com/questions/5031061/prove-that-a-monoid-is-trivial-using-some-specific-conditions/5031092#5031092
\lstset{language=Isabelle}
\begin{lstlisting}
theory IbookCh0
  imports Main EX
begin
end
\end{lstlisting}

\lstset{language=sys}
\begin{lstlisting}
theory IbookCh0
  imports Main EX
begin
end
\end{lstlisting}

\ignore{
lemma "evens": "\<exists> (n::nat) . 2*n > (k::nat)"
  by presburger

lemma "evens2": "\<exists> (n::nat) . 2*n > (k::nat)"
  using evens by auto

lemma "evens3": "\<exists> (n::nat) . 2*n > (k::nat)"
proof -
  have example:"2*(k+1) > k" 
    by simp
  show ?thesis 
    using example by blast
qed
}

